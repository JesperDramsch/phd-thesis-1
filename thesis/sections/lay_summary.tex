%% The following is a directive for TeXShop to indicate the main file
%%!TEX root = ../thesis.tex

\chapter{Lay Summary}

Hydraulic fracturing is used for producing hydrocarbons from low-permeability reservoirs. Fluid and proppant (commonly sand) are injected to create fractures and to keep those fracture pathways open. One of the unknowns in this process is the distribution of the injected materials. If the electrical conductivity of the injected materials is distinct from the reservoir rock, electromagnetic geophysical data, which are sensitive to the distribution of those materials, can be collected. Through an inversion process, a 3D model of the injected materials can be estimated from those data. The presence of steel-cased wells complicates this procedure because steel has vastly different electrical and magnetic properties than the reservoir rock. As a result, it significantly alters the behavior of the electromagnetic fields. This thesis examines aspects of the fundamental physics of electromagnetics in settings with steel-cased wells and as well as strategies for estimating the distribution of injected materials from electromagnetic data.


